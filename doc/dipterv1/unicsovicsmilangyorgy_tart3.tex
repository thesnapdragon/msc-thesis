\documentclass{article}
\usepackage[utf8x]{inputenc}
%\usepackage{a4wide}
\usepackage[magyar]{babel}
\usepackage{times}
\usepackage{graphicx}
\usepackage[top=0.5in, bottom=0.5in]{geometry}
%opening
\author{Milán Unicsovics}
\title{MSc Önálló laboratórium 2}
\date{\today}
\sloppy
\begin{document}
%%%%%%%%%%%%%%%%%%%Szöveg%%%%%%%%%%%%%%%%%%%%%%
\thispagestyle{empty}
\begin{figure}[htp]
\centering
\includegraphics[scale=0.3]{figures/bme_logo.pdf}
\begin{center}
Budapesti Műszaki és Gazdaságtudományi Egyetem\\
Méréstechnika és Információs Rendszerek Tanszék
\end{center}
\end{figure}
\vspace*{-0.1in}
\begin{center}
\subsection*{Tesztgenerálás állapotgép alapú modellekből}
{\bf
Unicsovics Milán György (II. évf, MSc) mérnök inf. szakos hallgató\\[0.3cm]

Konzulens: Dr. Micskei Zoltán adjunktus, MIT\\[0.3cm]

Szolgáltatásbiztos rendszertervezés szakirány

Diplomatervezés 1 összefoglaló

2014/15. II. félév
}
\end{center}
\vspace{0.5cm}

A tesztelés célja a hibadetektálás, mely során egy szoftver elvárt és aktuális működését összehasonlítjuk. A modell alapú tesztelés ennek egy változata, ahol modellekel írjuk le a tesztelni kívánt szoftver viselkedését. A modellekből tesztek generálhatóak, melyek később futtathatóak a szoftveren. Kutatásaim célja, hogy későbbiekben egy olyan eszköz fejlesztésébe kezdhessek, mely állapotgép alapú modellt használó szoftverekhez tesztelő keretrendszerként használható.

Az előző két félév feladata a modell alapú tesztelés elméleti hátterének és a modell alapú tesztelési folyamat megismerése volt. Ezen ismereteket az elérhető ipari tesztelő eszközökkel való ismerkedéssel folytattam. A tapasztalatok elmélyítését néhány példa implementáció elkészítésével fejeztem be.

Az idei félévet egy UML szemantikához hasonló állapotgép leíró modell, a CERN és a BME együttműködéséből Darvas Dániel által készített PLC-HSM leírónyelv megismerésével kezdtem. A formális, hierarchikus, moduláris nyelv könnyen felhasználható volt a tesztgenerálási folyamat részeként, így sikerült egy iparban is használt modell megoldást beépíteni a tervezés során.

Tesztgenerálási algoritmusként az előző félévek tapasztalatai alapján egy erősebb eszközt vetettem be. Ez az Alloy modellező nyelv volt, melynek kiismerésével a félévet folytattam. A nyelv képességeinek elsajátítását a modellellenőrzés világából vett verifikációs módszer implementálásával folytattam Alloy-t felhasználva.

Először a PLC-HSM metamodelljét kellett Alloy segítségével modellezni, majd ezt a modellt kiegészíteni a kívánt tesztesetek metamodelljével. A generált tesztesetek állapot-, vagy átmenetlefedettséget képesek garantálni az adott szoftver modelljét tekintve.

Az Alloy kódok felépítését megvizsgálva nyilvánvalóvá válik, hogy a példánymodell információit vizsgálva az Alloy kód nagy része automatikusan generálható. Ezt a generálást az Eclipse Modeling Tools részét képező Acceleo segítségével végeztem el. Az Acceleo valójában egy olyan eszköz, amely modelleket képes szöveg formátumba transzformálni, így jól illik a modellvezérelt fejlesztés módszertanába is.

A félévet az eddigi tapasztalatok és eredmények összefoglalásával folytattam, mely dokumentációk a diploma részeiként fognak szolgálni.

A diploma felépítésének megtervezése során a rendszer továbbfejlesztési lehetőségei is nyilvánvalóvá váltak. Mindenképpen fontos a rendszer teljesítményének vizsgálata, hogy a különböző méretű és felépítésű modelleken milyen korlátok között generálhatóak tesztesetek. Valószínűleg a modellek méretének növelésével a rendszer skálázódásának finomhangolására is szükség lesz, így használhatóak lesznek a korábbi félévek gráfalgoritmus implementációi is a kevert stratégiájú teszgenerálási algoritmusok használatakor. További lehetőségként szolgálhat a bonyolultabb állapotgép elemek támogatása.

Az elvégzett feladatok során az eddig összegyűjtött tapasztalatokra építve megkezdhettem a rendszer tervezését és implementációját. A tervezés során olyan eszközöket választottam, amelyek teljesítik a kitűzött célokat, majd az implementáció során sikerült egy végleges verzió létrehozását megkezdeni. A célként kitűzött alkalmazás jelenlegi változata, már megfelel az alapvető követelményeknek, csak finomhangolásra lesz szükség, illetve további szolgáltatásokkal lesz kiegészítve a rendszer. Összesítve elmondhatom, hogy nagyon élveztem a féléves feladatokat és sikerült egy olyan alkalmazást készíteni, mely teljesíti a kitűzött célokat és a következő félévben könnyen továbbfejleszthető lesz egy végleges verzióvá.

\end{document}
