%----------------------------------------------------------------------------
\chapter{Introduction}
\label{cha:introduction}
%----------------------------------------------------------------------------

\section{Problem and thesis statement}
\label{sec:problem}

The main goal of software testing is fault detection, where we compare the software's intended and actual behaviour to make sure there are not any difference between those, regarding the requirements.

These methods are usually very time and resource consuming activities. The process is often undocumented, unrepeatable and unstructured, that's why creating tests limited by the ingenuity of the single developer. Furthermore the traditional test cases are static and hard to update, but the software under test is dynamically evolving. One other problem of the handcrafted test is, that they suffer from "pesticide paradox". The test are getting less effective during the testing process, because the tester writes them with the same method for mostly solved problems.

Model-based testing substitutes the traditional ad-hoc software testing methods which relies on behaviour models that describe the intended behaviour of the system and its environment. From the models set of test cases are generated automatically and then executed on the tested software.

My research aims to prepare to create a new automated testing framework for software based on state machine models. Before that related works and similar solutions have to be examined. The result of the research later can be used to design and develop a software which fills the need of a fully automated model based testing framework.

% section problem (end)

\section{Proposed approach}
\label{sec:proposedapproach}

...

% section proposedapproach (end)

% chapter introduction (end)